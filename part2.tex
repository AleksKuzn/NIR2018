% вторая часть

\section{Обзор задач и решений в рамках умного университета}

\textbf{Электронный кампус}
Современные университеты – это маленькие города.\cite{Cisco} В них есть библиотеки, концертные залы, спортивные залы, бассейны, магазины, больницы, гостиницы, общежития, офисы, рестораны, столовые, парковки, аудитории, расчетные центры, платежные терминалы. В них есть жители – студенты, преподаватели и сотрудники, есть гости – абитуриенты, родители, работодатели, партнеры. Чтобы все это функционировало, чтобы для каждого жителя и гостя университета был доступ к ресурсам, службам и сервисам в соответствии с их ролью,  в университете необходимы:
 \begin{itemize}
	\item техническая инфраструктура - вычислительная сеть, включая оборудование беспроводного доступа, компьютерное оборудование, устройства телекоммуникации и связи, презентационное и видео оборудование, мобильные устройства для доступа к цифровым ресурсам,  системы контроля и управления доступом к ресурсам, системы сигнализации и видеонаблюдения
	\item информационная инфраструктура, реализованная в виде цифровых ресурсов, приложений и сервисов корпоративной информационной среды
	\item единый атрибут для доступа к ресурсам университета – персональные идентификационные карты  (типа proximity или smart)
\end{itemize}
Концепция электронного кампуса позволяет полнее раскрыть потенциал университета и оптимизировать имеющиеся в университете ресурсы. Реализованная в настоящее время в университете модель  цифрового  кампуса обеспечивает студентам:
\begin{itemize}
	\item доступ на территорию и  в общежития по  идентификационной пластиковой карте;
	\item доступ в Интернет и к цифровым ресурсам университета из любой точки кампуса через проводную или беспроводную сеть;
	\item доступ в библиотеку и к множественному образовательному контенту в форме текста, графики, видео и аудиоматериалов, презентаций к занятиям, видеолекций, тестов  и т.п.
	\item доступ к сервису видеоматериалов с использованием технологии потокового вещания;
	\item доступ к занятиям и консультациям из удаленных точек – через видеоконференцсвязь и вебинары,  что повышает мобильность студентов, обеспечивает общение с преподавателями и студентами, участниками партнерских программ университета;
	\item доступ к спортивным, медицинским услугам;
	\item доступ к сервисам портала университета – индивидуальный план обучения студента, расписание занятий, успеваемость, выполнение курсовых и дипломных работ, ведение студенческих  проектов, контроль платежей и т.д.
\end{itemize}
Для преподавателей и сотрудников цифровой кампус обеспечивает:
 \begin{itemize}
	\item доступ на территорию и в помещения  по  идентификационной пластиковой карте;
	\item доступ в Интернет и к цифровым ресурсам университета из любой точки кампуса через проводную или беспроводную сеть
	\item доступ в библиотеку
	\item возможность публиковать образовательный контент (тексты, презентации, видео);
	\item возможность проводить занятия для удаленных студентов  через видеоконференцсвязь, вебинары и возможность  вести занятия, находясь удаленно от кампуса
	\item доступ на автомобильную  парковку университета
	\item доступ к спортивным, медицинским услугам, а также к услугам службы питания
	\item возможность реализовывать свои бизнес-процессы по разным направлениям деятельности с помощью ИТ, добиваясь более качественных результатов, с меньшими затратами и с большей производительностью за счет
	\begin{itemize}
		\item автоматизации учета – студентов, сотрудников, аспирантов, выпускников, материальных ценностей, недвижимости, библиотечного фонда, образовательных ресурсов, научных проектов, посещаемости, ремонтов зданий и помещений, расходов, доходов, публикаций и много другого
		\item автоматизации расчетов – заработной платы, табелей, амортизации, нагрузки, рейтингов студентов, преподавателей, кафедры, стипендии, трафика, закупок, оплаты  за обучение, Интернет, проживание, услуги спорта,  медицины, службы питания и т.д.
		\item автоматизация процессов – формирование образовательных  программ и учебных планов, графиков учебного процесса, индивидуальных траекторий обучения, расписания занятий, приказов, договоров, сайтов, проведение сессий, практик, экзаменов, подачи и обработки заявок, планирование и отчетность деятельности подразделений, управления доступом к ресурсам, планирования ремонтов, расходов, доходов  и многое другое
		\item предоставления доступа к необходимой актуальной информации по всем направлениям деятельности университета
	\end{itemize}
\end{itemize}
В задачах управления университетом электронный кампус через сервисы КИС обеспечивает:
 \begin{itemize}
	\item Управление оргструктурой и персоналом
	\item Применение методов принятия решений на основе данных корпоративной информационной среды
	\item Использование агрегированных хранилищ данных и методов анализа данных
	\item Использование методов и технологий анализа бизнес-процессов (Business Intelligence)
	\item Применение системы управления электронного документооборотом для всех процессов, ориентированных на документы
	\item Контроль исполнительской дисциплины
	\item Планирование и отчетность по направлениям деятельности
\end{itemize}
Развитие инноваций в вузе (а информационные технологии являются осно­вой инноваций в управлении и организации учебного процесса), — это ключ к решению проблемы обеспечения высокой конкурентоспособности вуза. В большой степени успехи, достигнутые в этом направлении во ВГУЭС, обу­словлены тем, что коллектив университета доверяет решениям, которые при­нимаются ректоратом и реализуются ИТ-службой совместно с персоналом университета. Огромное значение для достижения цели имеет человеческий фактор.