%
% Шаблон для НИР
%

\documentclass[a4paper,12pt]{article}
\usepackage[backend=biber,sorting=none,style=gost-numeric,autolang=other]{biblatex} % библиография
%\usepackage[backend=biber,sorting=none,style=gost-numeric]{biblatex} % библиография
\usepackage{mathtext} %русские буквы в формулах
\usepackage[T2A]{fontenc}
\usepackage[utf8]{inputenc}
\usepackage[english,russian]{babel}
\usepackage{amsmath}
\usepackage{fancyvrb}
\usepackage{formular}
\usepackage{setspace} % управление междустрочными интервалами
%поля документа
\usepackage[left=3cm,right=1cm,top=2cm,bottom=2cm]{geometry}

\usepackage{misccorr} % точки в конце номеров разделов, использовать перед пакетом ccaption!
\usepackage{ccaption} % изменения подписей к рисункам и табл.

\usepackage[nooneline]{caption} 
\captionsetup[table]{justification=raggedright} % заголовок таблицы выравнивается влево
\captionsetup[figure]{justification=centering,labelsep=endash} % заголовок рисунка - по центру

% отступ перед первым абзацем
\usepackage{indentfirst}
%вставка изображений
\usepackage{graphicx}
% счетчики
\usepackage{totcount}
% управление содержанием
\usepackage{tocloft}
% управление таблицами и рисунками
\usepackage{float}

\newcounter{mycitecount}                                %% Счётчик библиографии
\AtEveryBibitem{\stepcounter{mycitecount}}              %% Работает для biblatex

\usepackage[figure,      %
            table,       %
            mycitecount, xspace ]{totalcount}           %% Подсчёт общего количества объектов в документе

% окружение для листингов - с нумерацией строк слева
\DefineVerbatimEnvironment{MyCode}{Verbatim}{frame=lines,numbers=left,numberblanklines=false,framesep=5mm}

% автоматическая нумерация листингов
\newfloat{Program}{phb}{lop}
\floatname{Program}{Листинг}
\floatstyle{ruled}

\setcounter{secnumdepth}{3} % глубина нумерации до подразделов

%если нужны точки в оглавлении для разделов - раскомментируйте следующую команду
%\renewcommand{\cftsecleader}{\cftdotfill{\cftdotsep}}

\addto\captionsrussian{%
\renewcommand{\figurename}{Рисунок}%
\renewcommand{\tablename}{Таблица}%
}

% дефис в подписи к рисункам
\captiondelim{ -- } 

% Настройки для окружений с подчеркиваниями для подписей и пр.
\setFRMfontencoding{T2A}
\setFRMdfontencoding{T2A}
% thanks to A.Starikov
\setFRMfontfamily{cmr}
\setFRMdfontfamily{ptm}
\setFRMdfontsize{10pt}

% задает длину поля для подписи на титульной странице
\newFRMfield{xtitlesign}{32mm}

% поле для факультета или кафедры
\newFRMfield{fcath}{65mm}

%имя файла с библиографией в формате BibTex
\addbibresource{rbiblio.bib}

\begin{document}

% счетчики страниц, рисунков, таблиц
\regtotcounter{page}
\regtotcounter{figure}
\regtotcounter{table}

\renewcommand{\refname}{\centerline{СПИСОК ИСПОЛЬЗОВАННОЙ ЛИТЕРАТУРЫ}} 
\renewcommand{\contentsname}{\centerline{СОДЕРЖАНИЕ}} 
%\renewcommand{\refname}{Список источников}  % По умолчанию "Список литературы" (article)
%\renewcommand{\bibname}{Литература}  % По умолчанию "Литература" (book и report)

% титульная страница
\thispagestyle{empty}
\begin{center} \small
\textbf{МИНИСТЕРСТВО ОБРАЗОВАНИЯ И НАУКИ РОССИЙСКОЙ ФЕДЕРАЦИИ}\\
ФЕДЕРАЛЬНОЕ ГОСУДАРСТВЕННОЕ АВТОНОМНОЕ ОБРАЗОВАТЕЛЬНОЕ УЧРЕЖДЕНИЕ
ВЫСШЕГО  ОБРАЗОВАНИЯ\\
«Национальный исследовательский ядерный университет «МИФИ»\\
\textbf{Обнинский институт атомной энергетики} – \\
филиал федерального государственного автономного образовательного учреждения высшего\\
образования «Национальный исследовательский ядерный университет «МИФИ»\\
(ИАТЭ НИЯУ МИФИ)
\end{center}
%\vfill
\medskip

% Направление подготовки следует уточнять,
% магистры и бакалавры могут иметь разные наименования
\begin{center}
\begin{tabular}{rl}
Отделение & \useFRMfield{fcath}[\large Интеллектуальные кибернетические системы] \\ 
Направление подготовки & \useFRMfield{fcath}[\large Информационные системы и технологии] \\ 
\end{tabular} 
\end{center}

\vfill

\large 

\begin{center}
	Научно-исследовательская работа \\
	
	\medskip
	
	\textbf{\Large 
		 Контроль электроэнергии и водоснабжения в рамках  умного города
	}
	
\end{center}

\vspace{1cm}

\begin{tabular*}{\textwidth}{lcr}
Студент группы ИС-М18 & \useFRMfield{xtitlesign} & А.В.Кузнецов\\
& & \\
Руководитель & & \\
д.т.н., профессор ОИКС & \useFRMfield{xtitlesign} & Б.И.Яцало
\end{tabular*}


\vfill
\large

\begin{center}
Обнинск, 2018 г
\end{center}

\onehalfspacing

\pagebreak

% реферат
\thispagestyle{empty}

\section*{\centering РЕФЕРАТ}

% возможно, кол-во источников придется вставлять вручную
\total{page} стр., \total{table} табл., \total{figure} рис. , \totalmycitecounts ист. 

УМНЫЙ ГОРОД, УМНЫЙ УНИВЕРСИТЕТ, КОНТРОЛЬ РЕСУРСОВ, ИНТЕРНЕТ ВЕЩЕЙ

Настоящая работа посвящена изучению перспектив создания <<Умного>> города, обзору задач и решений в рамках умного университета и разработке алгоритма поиска проблемных счетчиков электроэнергии и водоснабжения для разных типов ошибок.

Разработанная система учета ресурсов дает возможность жильцам контролировать свои расходы.

\pagebreak
\thispagestyle{empty}

\section*{\centering ОПРЕДЕЛЕНИЯ}
«Умный город» --- концепция интеграции нескольких информационных и коммуникационных технологий (ИКТ) и Интернета вещей (IoT решения) для управления городским имуществом
Интернете вещей (Internet of Things, IoT) --- физические объекты, такие, как устройства, датчики и системы, могут самостоятельно, без вмешательства человека, посылать и принимать данные через Интернет, взаимодействуя друг с другом или с внешней средой. 

\pagebreak

\section*{\centering ОБОЗНАЧЕНИЯ И СОКРАЩЕНИЯ}
АРМ --- автоматизированное рабочее место\\
ПУ --- приборы учета\\
ПО --- программное обеспечение\\
ГВС --- Горячее водоснабжение.\\
ХВС --- Холодное водоснабжение.\\



\pagebreak



\tableofcontents
% если нужно добавить "Стр." над номерами страниц - раскомментируйте следующую команду
%\addtocontents{toc}{~\hfill\textbf{Стр.}\par}

\pagebreak

\section*{\centering ВВЕДЕНИЕ}
\addcontentsline{toc}{section}{ВВЕДЕНИЕ}
% пример введения

В современных условиях разработка и реализация концепции <<умного города>> остается одним из главных направлений развития городов в индустриально развитых странах. Это наиболее явно проявляется в странах, столкнувшихся с целым спектром инфраструктурных и социальных проблем.

В России, где три четверти населения проживает в городах, внедрение технологий, стимулирующих экономику, улучшение управления городскими системами и качества жизни должно быть одной из наиболее актуальных задач. Новые технологии, наряду с модернизацией инфраструктуры, могут способствовать устранению технологической отсталости российских городов, а использование интеллектуальных систем может создать основу устойчивого развития.\cite{Harrison}
Цель данного исследования заключается в изучении перспектив создания <<умных городов>> в России на основе выявления проблем и возможных методов их невилирования при разработке и реализации концепции <<умных городов>

Основным результатом выполнения проекта будет программное обеспечение, предназначенное для считывания, мониторинга и работы с данными домовых счетчиков учета ресурсов.

Задачи, решаемые в ходе работы (в соответствии с заданием на НИР):
 \begin{enumerate}
 	\item Обзор проблематики умного города
	\item Обзор задач и решений в рамках умного университета 
	\item Примеры умных городов
	\item Алгоритмы поиска проблемных счетчиков электроэнергии и водоснабжения для разных типов ошибок
	\item Описание программы для управления ресурсами электроэнергии и водоснабжения
\end{enumerate}
 % текст введения в файле intro.tex
\pagebreak

%\input{Post_zad}
\pagebreak
% первая часть

%\section{Обзор проблематики умного города}
%http://statref.ru/ref_jgemeryfsotrotr.html
\section{Что такое умный город или “SmartCity”}

\subsection{Принципы Умного города}

Термин «умный город» появился относительно недавно, и определенного конкретного определения этому понятию нет. Но все-таки  эксперты сошлись в том, что главный источник управления смарт сити – данные о населении. 

Умный город (smart city) - это стратегическая концепция по развитию городского пространства, подразумевающая совместное использование информационно - коммуникационных технологий (ИКТ) и решений Интернета вещей (IoT) для управления городской инфраструктурой. К нему относятся транспортные системы, водопроводные каналы, медицинские организации, системы переработки отходов и множество других общественных служб. 

Главная идея системы «Умный город» - организация информационного пространства, которое содержит в себе данные о работе контролируемых объектов (счетчиков тепловой и электрической энергии, лифтов, электротехнического оборудования, различных технических средств безопасности и т.д.). На любом расстоянии можно управлять объектами в режиме реального времени, вне зависимости от места расположения объектов и центрального управляющего пункта в городе.\cite{NK}

Найти слабые места в работе организации, поставщиков ресурсов, оборудования и персонала можно проанализировав данные. Введение в эксплуатацию системы «Умный город» позволяет не только контролировать работу оборудования, но и принимать максимально верные управленческие решения. 

Для целостной системы «Умный город» должны быть функционально законченные подсистемы:
\begin{itemize}
 \item диспетчеризации и контроля лифтов; 
 \item автоматизированного коммерческого контроля и учета энергоресурсов и электроэнергии; 
 \item охранно-пожарной сигнализации и видеонаблюдения; 
 \item контроля доступа в помещения и к оборудованию; 
 \item управления оборудованием и инженерными сооружениями; 
 \item другие дополнительные системы, такие как контроль затопления подвалов, сигнализация загазованности горючими газами, экстренной голосовой связи
\end{itemize}
 
  Основные принципы Умного города (Smart City): 
  \begin{itemize}
  \item Микрорайон как градостроительная единица 
  \item Автономность города Социальная, деловая и  культурная самодостаточность 
  \item Разработка по стандартам экологичного строительства 
  \item Использование новейших информационных и коммуникационных технологий 
  \item Внедрение инновационных технологий энергетики, транспорта и строительства 
\end{itemize}

  Основными механизмами правильной организации и  оптимизации использования ресурсов в «умном» городе являются: 
  \begin{itemize}
  \item снижение неравномерности потребления в период пиков и провалов (основная проблема всех инфраструктур) т.е. распределение нагрузки на инфраструктурные сети во времени; 
  \item создание сетевых, а не линейных систем поставки ресурса позволят маневрировать потоками и «обходить» аварийные или пиковые участки, т.е. распределение в пространстве; 
  \item создание динамически управляемых источников мощности: накопители, демпферы, малоинерционные генераторы, и др.; 
  \item создание распределенной генерации различного масштаба;
  \item снижение потерь и ресурсопотребления конечных пользователей («умные» дома, энергоэффективное оборудование и др.)
\end{itemize}

\subsection{Актуальные проблемы умных городов}
Рассматривая возможные негативные аспекты умных городов, надо иметь в виду несколько нюансов.\cite{Almanah} Конечно же, проблема приватности к ним относится. Мы даже не осознаем, насколько огромные объемы данных о нас постоянно создаются и сохраняются. Google знаком с нашими поисковыми привычками в интернете, с нашим отношением к покупкам, с нашей персональной информацией и так далее. Камеры по всему городу отслеживают перемещения людей и машин, высматривают угрозы для безопасности, происшествия и прочие аномалии. Участвовать или не участвовать в наиболее персональных аспектах умных городов и сборе информации об индивидах должно быть личным делом каждого. Также должны существовать надежные способы шифрования и последующего уничтожения информации об индивидах, чтобы гарантировать их право на личную жизнь.

Инвестиции в умные города и расходы на их методы функционирования и технологии высоки на начальном этапе, но с течением времени эти расходы превращаются в долгосрочную экономию. Это означает, что политики, избранные всего на несколько лет, вполне могут не быть заинтересованы в инвестициях в долгосрочные стратегии. Поэтому должен существовать консенсус между политиками и их электоратом относительно того, в каком направлении движется развитие города, каковы его приоритеты и во что нужно инвестировать.\cite{Innoprom}

Координация — вот еще один важный аспект для внедрения передовых практик в умный город. Чтобы городские службы были эффективны, их необходимо координировать. Зачастую технологии или различные услуги предоставляются разными компаниями. Эти компании должны понимать, что в их интересах и в интересах жителей города работать вместе, а не просто продавать как можно больше товаров или услуг. Городские управленцы должны понимать концепцию умного города и делать так, чтобы скоординированные инвестиции направлялись на достижение наилучшего желаемого результата.


\subsection{Пример Умного города}

В штате Колорадо есть город Боулдер население которого около 100 тыс. человек. Этот город стал первым «умным городом» в мире. Жители этого замечательного города сразу же приобрели репутацию «хранителей» окружающей среды. Они применяют новейшие технологии, которые на много меньше вредны нашей природе. Дома жителей Боулдера наполнены самыми последними экологичными и энергосберегающими устройствами: панелями солнечных батарей, электрическими автомобилями и специализированными системами обогрева, охлаждения и освещения, которые объединяются единой системой мониторинга, сообщающей домовладельцу данные об углеродном следе дома (то есть количестве CO2, поступающего в атмосферу). Рей Гогель из Xcel Energy, член сервисной компании, которая внедрила новую систему, сказал: «Нам нравится думать об умной электросети как о примирении мира Томаса Эдисона с миром Билла Гейтса. Мы делаем то, в чем нуждаются все». В этом городе определенные дома имеют возможность самостоятельно управлять энергосберегающими технологиями. 

Очень актуальными являются солнечные батареи и интеллектуальные счетчики. Контролировать потребление энергии достаточно легко благодаря новой системе. Управление автомобилем, гаражом, домом можно осуществлять через компьютер. Все части дома взаимосвязаны и общаются между собой. Все эти технологии очень экономичны и сильно упрощают жизнь людям. Есть такие горожане у которых счет за коммунальные услуги составляют всего 3 доллара в месяц. 

Очень удобной стала покупка автомобиля можно выбрать его из списка по дате, цене или названию авиакомпании. Все можно сделать онлайн! Используя новую энергосистему можно решить, использовать возобновляемые источники или по-прежнему выбрать энергию сжигаемого угля. Для того, чтобы принять решение не обязательно находиться в данном городе, достаточно выйти в Интернет, чтобы проследить за всеми процессами и управлять ими. Теперь не страшно оставить включенным какое-либо устройство, просто достаточно зайти через сеть и выключить. 

Такие многофункциональные системы- это наше будущее. 

В скором времени интеллектуальные сети станут стандартом для всех новых домов. Каждый человек конечно же хочет минимально использовать энергию.

\subsection{Российские проекты умных городов}
В России за последние годы появился ряд проектов по интеллектуализации городских сервисов. В основном это умное управление
в трех сферах: электроэнергия, транспортные потоки, общественная
безопасность.\cite{smartcity}.

\subsubsection{ЦОД «Омский»}
Вокруг центра обработки данных «Омский» предполагается разместить промышленно-логистический парк (IT-кластер, высокотехнологичный сельскохозяйственный комплекс, производственные и лабораторные помещения), объекты жилой и социальной недвижимости (зона жилой застройки на 14 тыс. человек), административно-деловой центр и другие
объекты. Технологии умного города появятся в разработках самих резидентов кластера – IT-компаний и стартапов в сфере облачных технологий. Подобно японским городам, «Омский» ставит целью перейти
на самоокупаемость, энерго- и даже отчасти продуктовую независимость: например, самостоятельно генерировать электричество, а его избытки передавать на содержание теплиц.

\subsubsection{Ильинское-Усово}
Проектировщики ЖК «Ильинское-Усово» внедряют умные городские технологии уже на первой очереди строительства микрорайона: это новые материалы, Интернет вещей в части ЖКХ, безопасности и транспорта. ГК «Мортон» определяет 15 направлений развития умного города, и отдает предпочтение решениям, которые охватывают сразу несколько из них. На фоне базовых технологий (интеграция транспортной и инженерной систем,
энергосбережение, видеоаналитика и пр.) выделяется уникальная для России система недорогих решений в области носимой медицины. Сама «Мортон» инвестирует в ряд городских технологий, которые впоследствии тиражирует.
\subsubsection{Владивосток}
Во Владивостоке работает Единая дежурная диспетчерская служба по управлению транспортом в реальном времени, а также автоматизированная система управления уличным освещением. «Ростелеком» внедряет в городе систему экстренного вызова оперативных служб и модернизированную систему оповещения.
\subsubsection{Белгород}
В Белгороде установлены датчики на распределительных сетях, которые минимизируют последствия аварий, и умное освещение.
\subsubsection{Екатеринбург}
Екатеринбург планирует создать интеллектуальную энергосеть к 2030 году. Столько займет модернизация существующих и построение новых энергообъектов с учетом требований Smart Grid, в том числе внедрение транспортных средств на электротяге, перевод объектов – потребителей электроэнергии в режим ее генерации.

\subsection{"Умный" город будет создан в Обнинске}
В Обнинске в рамках федеральной программы <<Цифровая экономика РФ>> будет создан "умный" город. Об этом было заявлено калужским губернатором Анатолием Артамоновым на инвестиционном форуме в Сочи в феврале 2018 года.\cite{smartcityObn}
Выступая на сессии форума, калужский губернатор Анатолий Артамонов акцентировал внимание собравшихся, что для реализации данных проектов необходимо, в первую очередь, совершенствовать инфраструктуру. Прежде всего, это касается городов, где она формировалась давно. По мнению губернатора, решение данной задачи возможно только при условии реализации единой госполитики. 

"Это повлечет за собой большие расходы, но данное направление необходимо включить в число приоритетных", - предложил Артамонов. Что касается Обнинска, то создание "умного города" входит в стратегию социально-экономического развития наукограда на ближайшие 9 лет. 

"Тема новая, но очень перспективная. Нам нужно стремиться к тому, чтобы в будущем не только крупные города, но и небольшие населенные пункты могли внедрить у себя "умные" технологии", - отметил губернатор. 

Ключевыми направлениями «Умного города», охватывающими все виды социально-экономической деятельности городов, являются: «Умная экономика», «Умная мобильность», «Умная среда», «Умные люди», «Умная безопасность», «Умная медицина», «Умное проживание» и «Умное управление». 

В целом, проект можно считать достаточно молодым. Находится он на стадии старта. Но уже успел завоевать доверие и признанность не в одной стране мира. Масштаб и продвижение проекта в разных точках мира зависят, в основном, от времени начала его ввода в пользование. 

Каждый житель города имеет право принимать участие в развитии проекта, предлагать идеи и привлекать интерес, к таким деталям, на которые, по его мнению, стоит обратить внимание.   % первая глава - в файле part1.tex
\pagebreak
% вторая часть

\section{Обзор задач и решений в рамках умного университета}

\textbf{Электронный кампус}
Современные университеты – это маленькие города.\cite{Cisco} В них есть библиотеки, концертные залы, спортивные залы, бассейны, магазины, больницы, гостиницы, общежития, офисы, рестораны, столовые, парковки, аудитории, расчетные центры, платежные терминалы. В них есть жители – студенты, преподаватели и сотрудники, есть гости – абитуриенты, родители, работодатели, партнеры. Чтобы все это функционировало, чтобы для каждого жителя и гостя университета был доступ к ресурсам, службам и сервисам в соответствии с их ролью,  в университете необходимы:
 \begin{itemize}
	\item техническая инфраструктура - вычислительная сеть, включая оборудование беспроводного доступа, компьютерное оборудование, устройства телекоммуникации и связи, презентационное и видео оборудование, мобильные устройства для доступа к цифровым ресурсам,  системы контроля и управления доступом к ресурсам, системы сигнализации и видеонаблюдения
	\item информационная инфраструктура, реализованная в виде цифровых ресурсов, приложений и сервисов корпоративной информационной среды
	\item единый атрибут для доступа к ресурсам университета – персональные идентификационные карты  (типа proximity или smart)
\end{itemize}
Концепция электронного кампуса позволяет полнее раскрыть потенциал университета и оптимизировать имеющиеся в университете ресурсы. Реализованная в настоящее время в университете модель  цифрового  кампуса обеспечивает студентам:
\begin{itemize}
	\item доступ на территорию и  в общежития по  идентификационной пластиковой карте;
	\item доступ в Интернет и к цифровым ресурсам университета из любой точки кампуса через проводную или беспроводную сеть;
	\item доступ в библиотеку и к множественному образовательному контенту в форме текста, графики, видео и аудиоматериалов, презентаций к занятиям, видеолекций, тестов  и т.п.
	\item доступ к сервису видеоматериалов с использованием технологии потокового вещания;
	\item доступ к занятиям и консультациям из удаленных точек – через видеоконференцсвязь и вебинары,  что повышает мобильность студентов, обеспечивает общение с преподавателями и студентами, участниками партнерских программ университета;
	\item доступ к спортивным, медицинским услугам;
	\item доступ к сервисам портала университета – индивидуальный план обучения студента, расписание занятий, успеваемость, выполнение курсовых и дипломных работ, ведение студенческих  проектов, контроль платежей и т.д.
\end{itemize}
Для преподавателей и сотрудников цифровой кампус обеспечивает:
 \begin{itemize}
	\item доступ на территорию и в помещения  по  идентификационной пластиковой карте;
	\item доступ в Интернет и к цифровым ресурсам университета из любой точки кампуса через проводную или беспроводную сеть
	\item доступ в библиотеку
	\item возможность публиковать образовательный контент (тексты, презентации, видео);
	\item возможность проводить занятия для удаленных студентов  через видеоконференцсвязь, вебинары и возможность  вести занятия, находясь удаленно от кампуса
	\item доступ на автомобильную  парковку университета
	\item доступ к спортивным, медицинским услугам, а также к услугам службы питания
	\item возможность реализовывать свои бизнес-процессы по разным направлениям деятельности с помощью ИТ, добиваясь более качественных результатов, с меньшими затратами и с большей производительностью за счет
	\begin{itemize}
		\item автоматизации учета – студентов, сотрудников, аспирантов, выпускников, материальных ценностей, недвижимости, библиотечного фонда, образовательных ресурсов, научных проектов, посещаемости, ремонтов зданий и помещений, расходов, доходов, публикаций и много другого
		\item автоматизации расчетов – заработной платы, табелей, амортизации, нагрузки, рейтингов студентов, преподавателей, кафедры, стипендии, трафика, закупок, оплаты  за обучение, Интернет, проживание, услуги спорта,  медицины, службы питания и т.д.
		\item автоматизация процессов – формирование образовательных  программ и учебных планов, графиков учебного процесса, индивидуальных траекторий обучения, расписания занятий, приказов, договоров, сайтов, проведение сессий, практик, экзаменов, подачи и обработки заявок, планирование и отчетность деятельности подразделений, управления доступом к ресурсам, планирования ремонтов, расходов, доходов  и многое другое
		\item предоставления доступа к необходимой актуальной информации по всем направлениям деятельности университета
	\end{itemize}
\end{itemize}
В задачах управления университетом электронный кампус через сервисы КИС обеспечивает:
 \begin{itemize}
	\item Управление оргструктурой и персоналом
	\item Применение методов принятия решений на основе данных корпоративной информационной среды
	\item Использование агрегированных хранилищ данных и методов анализа данных
	\item Использование методов и технологий анализа бизнес-процессов (Business Intelligence)
	\item Применение системы управления электронного документооборотом для всех процессов, ориентированных на документы
	\item Контроль исполнительской дисциплины
	\item Планирование и отчетность по направлениям деятельности
\end{itemize}
Развитие инноваций в вузе (а информационные технологии являются осно­вой инноваций в управлении и организации учебного процесса), — это ключ к решению проблемы обеспечения высокой конкурентоспособности вуза. В большой степени успехи, достигнутые в этом направлении во ВГУЭС, обу­словлены тем, что коллектив университета доверяет решениям, которые при­нимаются ректоратом и реализуются ИТ-службой совместно с персоналом университета. Огромное значение для достижения цели имеет человеческий фактор. % вторая глава - в файле part2.tex
\pagebreak

% если есть еще разделы - сохраните их в соответствующих файлах и раскомментируйте строки ниже, при необходимости добавьте еще
% третья часть

\section{Система учета ресурсов}
Для каждой квартиры подключаются импульсные счетчики воды, тепла и электричества. Информация с них поступает на сервер. Анализирует на предмет неисправностей и утечек. И выводится на сайте, где житель может контролировать свои расходы и делать платежи.

Система будет сопровождаться программным обеспечением.
\subsection{Описание программы}
Система учета ресурсов предназначена для считывания, мониторинга и работы с данными домовых счетчиков учета ресурсов. Данные со счетчиков попадают на сервер баз данных, в программе диспетчера и администратора данные отображаются, считаются, формируются отчеты.

Полный функционал программного обеспечения:
\begin{itemize}
\item управление адресами и объектами установки ПУ;
\item управление приборами учета;
\item управление абонентами; 
\item просмотр показаний ПУ за выбранный интервал времени; 
\item расчет потребления энергоресурсов по основным показаниям ПУ за указанный интервал времени; 
\item просмотр детальной информации по потреблению энергоресурсов конкретного ПУ с выводом графика потребления; 
\item предоставление сведений об аварийных и нештатных ситуациях ПУ; 
\item экспорт полученных данных в другие форматы, вывод на печать;
\item поиск.
\end{itemize}

Система должна быть защищенной и поэтому используются локальные данные. Сервер, база данных на пост грессе и программы на компьютерах администраторов для управления. 

\textbf{Цели:} Сбор данных и мониторинг потребления ресурсов.
 
\textbf{Задачи:}  Создать защищенную и удобную систему  для сбора данных потребляемых населением.

Программа для администратора: мониторинг, добавление/ изменение/ удаление счетчиков, составление отчетов. 

Программа для диспетчера:  мониторинг, составление отчетов. 


\subsection{Алгоритмы поиска проблемных счетчиков электроэнергии и водоснабжения для разных типов ошибок}
Рассмотрим проблему, когда от счетчика не поступает импульс. Есть два варианта:
\begin{itemize}
	\item счетчик не исправен, по какой-то причине не крутится роликовый индикатор;
	\item счетчик работает, но импульсы от него не поступают.
\end{itemize}

Ошибка №1. Проблемы со счетчиком ХВС. 

Потребление ХВС = 0 за сутки, при этом потребление ГВС > 0 за сутки.

Алгоритм поиска:

Находим разность между показаниями счетчиков ХВС и ГВС, за текущее число и за вчерашнее.
Оставляем значения счетчиков ХВС и ГВС, где разность ХВС = 0.
Убираем значения счетчиков ХВС и ГВС, где разность ГВС = 0.

Оставшиеся счетчики ХВС являются проблемными (можно сравнивать не за сутки, а за 3 или 5).

Ошибка №2. Проблемы со счетчиком ГВС.

Потребление ГВС = 0 за неделю, при этом потребление ХВС > 0 за неделю.

Алгоритм поиска:

Находим разность между показаниями счетчиков ХВС и ГВС неделю назад и за текущее число.
Оставляем значения счетчиков ХВС и ГВС, где разность ГВС = 0.
Убираем значения счетчиков ХВС и ГВС, где разность ХВС = 0.

Получили счетчики, которые попадают в поле подозрения.
Нужно провести отбор, возможно горячей водой просто не пользуются.

Убираем счетчики, у которых значения ГВС < 5 и значение ХВС < 10.
Оставшиеся счетчики ГВС являются проблемными и требуют ручной проверки.

Ошибка №3. Проблемы со счетчиком ХВС и ГВС.

Потребление ХВС = 0, ГВС = 0 за неделю, при этом потребление Э  >= Ср.Ар. за предыдущую неделю.

Алгоритм поиска:

Находим разность между показаниями счетчиков ХВС и ГВС неделю назад и за текущее число.
Оставляем значения счетчиков ХВС и ГВС, где разность ГВС = 0 и ХВС = 0.
Находим среднее арифметическое потребление электричества за предыдущую неделю.
Если потребление электроэнергии за неделю больше чем среднее арифметическое значение, то счетчики попадают в список проблемных счетчиков.

Ошибка №4. Проблемы с электросчетчиком.

Потребление Электричества = 0 за сутки, при этом потребление ХВС > 0 и/или ГВС > 0 за сутки и более.

  % третья глава - в файле part3.tex
\pagebreak

%\input{part4} % четвертая глава - в файле part4.tex
%\pagebreak

%\input{part5}  % пятая глава - в файле part5.tex
%\pagebreak

\section*{\centering ЗАКЛЮЧЕНИЕ}
\addcontentsline{toc}{section}{ЗАКЛЮЧЕНИЕ}
В России интерес к тематике умного города растет с каждым годом, в том числе потому, что многие города подходят к пределам надежности и функциональности существующей инфраструктуры.

По результатам исследования были выделены три группы проблем при разработке и реализации концепции «умного города» в России.

Рассмотрены задачи и решения в рамках умного университета. 

Описана программа для управления ресурсами электроэнергии и водоснабжения.

Рассмотрены основные алгоритмы поиска проблемных счетчиков электроэнергии и водоснабжения для разных типов ошибок.

% оформление библиографии - вариант с БД
\pagebreak

\addcontentsline{toc}{section}{СПИСОК ИСПОЛЬЗОВАННОЙ ЛИТЕРАТУРЫ}
% ВАЖНО: для корректного отображения в списке литературы ссылок на англ.языке в bibtex-описание источника следует добавить поле 
% langid = {english}
\printbibliography

\end{document}          

