% пример введения

В современных условиях разработка и реализация концепции <<умного города>> остается одним из главных направлений развития городов в индустриально развитых странах. Это наиболее явно проявляется в странах, столкнувшихся с целым спектром инфраструктурных и социальных проблем.

В России, где три четверти населения проживает в городах, внедрение технологий, стимулирующих экономику, улучшение управления городскими системами и качества жизни должно быть одной из наиболее актуальных задач. Новые технологии, наряду с модернизацией инфраструктуры, могут способствовать устранению технологической отсталости российских городов, а использование интеллектуальных систем может создать основу устойчивого развития.\cite{Harrison}
Цель данного исследования заключается в изучении перспектив создания <<умных городов>> в России на основе выявления проблем и возможных методов их невилирования при разработке и реализации концепции <<умных городов>

Основным результатом выполнения проекта будет программное обеспечение, предназначенное для считывания, мониторинга и работы с данными домовых счетчиков учета ресурсов.

Задачи, решаемые в ходе работы (в соответствии с заданием на НИР):
 \begin{enumerate}
 	\item Обзор проблематики умного города
	\item Обзор задач и решений в рамках умного университета 
	\item Примеры умных городов
	\item Алгоритмы поиска проблемных счетчиков электроэнергии и водоснабжения для разных типов ошибок
	\item Описание программы для управления ресурсами электроэнергии и водоснабжения
\end{enumerate}
