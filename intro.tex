% пример введения

В современных условиях разработка и реализация концепции Умного города остается одним из главных направлений развития городов в индустриально развитых странах. Это наиболее явно проявляется в странах, столкнувшихся с целым спектром инфраструктурных и социальных проблем.

В России, где три четверти населения проживает в городах, внедрение технологий, стимулирующих экономику, улучшение управления городскими системами и качества жизни должно быть одной из наиболее актуальных задач. Новые технологии, наряду с модернизацией инфраструктуры, могут способствовать устранению технологической отсталости российских городов, а использование интеллектуальных систем может создать основу устойчивого развития.\cite{Harrison}
Цель данного исследования заключается в изучении перспектив создания Умных городов в России на основе выявления проблем и возможных методов их невилирования при разработке и реализации концепции Умных городов

Основным результатом выполнения проекта будет программное обеспечение, предназначенное для считывания, мониторинга и работы с данными домовых счетчиков учета ресурсов.

Задачи, решаемые в ходе работы (в соответствии с заданием на НИР):
 \begin{enumerate}
 	\item обзор проблематики Умного города;
	\item обзор задач и решений в рамках Умного университета; 
	\item примеры Умных городов;
	\item алгоритмы поиска проблемных счетчиков электроэнергии и водоснабжения для разных типов ошибок;
	\item описание программы для управления ресурсами электроэнергии и водоснабжения.
\end{enumerate}
