% первая часть

%\section{Обзор проблематики Умного города}
%http://statref.ru/ref_jgemeryfsotrotr.html
\section{Что такое Умный город или “SmartCity”}

\subsection{Принципы Умного города}

Термин Умный город появился относительно недавно, и определенного конкретного определения этому понятию нет. Но, все-таки  эксперты сошлись в том, что главный источник управления <<смарт сити>> – данные о населении. 

Умный город (smart city) - это стратегическая концепция по развитию городского пространства, подразумевающая совместное использование информационно - коммуникационных технологий (ИКТ) и решений Интернета вещей (IoT) для управления городской инфраструктурой. К нему относятся транспортные системы, водопроводные каналы, медицинские организации, системы переработки отходов и множество других общественных служб. 

Главная идея системы Умный город - организация информационного пространства, которое содержит в себе данные о работе контролируемых объектов (счетчиков тепловой и электрической энергии, лифтов, электротехнического оборудования, различных технических средств безопасности и т.д.). На любом расстоянии можно управлять объектами в режиме реального времени, вне зависимости от места расположения объектов и центрального управляющего пункта в городе.\cite{NK}

Найти слабые места в работе организации, поставщиков ресурсов, оборудования и персонала, а так же можно проанализировать данные. Введение в эксплуатацию системы Умный город позволяет не только контролировать работу оборудования, но и принимать максимально верные управленческие решения. 

Для целостной системы Умный город должны быть функционально законченные подсистемы:
\begin{itemize}
 \item диспетчеризации и контроля лифтов; 
 \item автоматизированного коммерческого контроля и учета энергоресурсов и электроэнергии; 
 \item охранно-пожарной сигнализации и видеонаблюдения; 
 \item контроля доступа к помещению и к оборудованию; 
 \item управления оборудованием и инженерными сооружениями; 
 \item другие дополнительные системы, такие как контроль затопления подвалов, сигнализация загазованности горючими газами, экстренной голосовой связи.
\end{itemize}
 
  Основные принципы Умного города (Smart City): 
  \begin{itemize}
  \item микрорайон как градостроительная единица; 
  \item автономность города социальная, деловая и  культурная самодостаточность;
  \item разработка по стандартам экологичного строительства; 
  \item использование новейших информационных и коммуникационных технологий;
  \item внедрение инновационных технологий энергетики, транспорта и строительства;
\end{itemize}

  Основными механизмами правильной организации и  оптимизации использования ресурсов в Умном городе являются: 
  \begin{itemize}
  \item снижение неравномерности потребления в период пиков и провалов (основная проблема всех инфраструктур), т.е. распределение нагрузки на инфраструктурные сети во времени; 
  \item создание сетевых, а не линейных систем поставки ресурса позволят маневрировать потоками и «обходить» аварийные или пиковые участки, т.е. распределение в пространстве; 
  \item создание динамически управляемых источников мощности: накопители, демпферы, малоинерционные генераторы и др.; 
  \item создание распределенной генерации различного масштаба;
  \item снижение потерь и ресурсопотребления конечных пользователей (Умные дома, энергоэффективное оборудование и др.).
\end{itemize}

\subsection{Актуальные проблемы Умных городов}
Рассматривая возможные негативные аспекты Умных городов, надо иметь в виду несколько нюансов.\cite{Almanah} Конечно же, проблема приватности к ним относится. Мы даже не осознаем, насколько огромные объемы данных о нас постоянно создаются и сохраняются. Google знаком с нашими поисковыми привычками в интернете, с нашим отношением к покупкам, с нашей персональной информацией и т.д. Камеры по всему городу отслеживают перемещения людей и машин, высматривают угрозы безопасности, происшествия и прочие аномалии. Участвовать или не участвовать в наиболее персональных аспектах Умных городов и сборе информации о людях должно быть личным делом каждого. Также должны существовать надежные способы шифрования и последующего уничтожения информации об индивидах, чтобы гарантировать их право на личную жизнь.

Инвестиции в Умные города и расходы на их методы функционирования и технологии высоки на начальном этапе, но с течением времени эти расходы превращаются в долгосрочную экономию. Это означает, что политики, избранные всего на несколько лет, вполне могут не быть заинтересованы в инвестициях в долгосрочные стратегии. Поэтому должен существовать консенсус между политиками и их электоратом относительно того, в каком направлении движется развитие города, каковы его приоритеты и во что нужно инвестировать.\cite{Innoprom}

Координация — вот еще один важный аспект для внедрения передовых практик в Умный город. Чтобы городские службы были эффективны, их необходимо координировать. Зачастую технологии или различные услуги предоставляются разными компаниями. Эти компании должны понимать, что в их интересах и в интересах жителей города работать вместе, а не просто продавать как можно больше товаров или услуг. Городские управленцы должны понимать концепцию Умного города и делать так, чтобы скоординированные инвестиции направлялись на достижение наилучшего желаемого результата.


\subsection{Пример Умного города}

В штате Колорадо есть город Боулдер население которого около 100 тыс. человек. Этот город стал первым Умным городом в мире. Жители этого замечательного города сразу же приобрели репутацию «хранителей» окружающей среды. Они применяют новейшие технологии, которые намного меньше вредны для нашей природы. Дома жителей Боулдера наполнены самыми последними экологичными и энергосберегающими устройствами: панелями солнечных батарей, электрическими автомобилями и специализированными системами обогрева, охлаждения и освещения, которые объединяются единой системой мониторинга, сообщающей домовладельцу данные об углеродном следе дома (то есть количество CO2, поступающего в атмосферу). Рей Гогель из Xcel Energy, член сервисной компании, которая внедрила новую систему, сказал: «Нам нравится думать об Умной электросети как о примирении мира Томаса Эдисона с миром Билла Гейтса. Мы делаем то, в чем нуждаются все». В этом городе определенные дома имеют возможность самостоятельно управлять энергосберегающими технологиями. 

Очень актуальными являются солнечные батареи и интеллектуальные счетчики. Контролировать потребление энергии достаточно легко благодаря новой системе. Управление автомобилем, гаражом, домом можно осуществлять через компьютер. Все части дома взаимосвязаны и общаются между собой. Все эти технологии очень экономичны и сильно упрощают жизнь людям. Есть такие горожане у которых счет за коммунальные услуги составляют всего 3 доллара в месяц. 

Очень удобной стала покупка автомобиля, можно выбрать его из списка по дате, цене или названию автосалона. Все можно сделать онлайн! Используя новую энергосистему можно решить, использовать возобновляемые источники или по-прежнему выбрать энергию сжигаемого угля. Для того, чтобы принять решение, не обязательно находиться в данном городе, достаточно выйти в Интернет, чтобы проследить за всеми процессами и управлять ими. Теперь не страшно оставить включенным какое-либо устройство, просто достаточно зайти через сеть и выключить. 

Такие многофункциональные системы --- это наше будущее. 

В скором времени интеллектуальные сети станут стандартом для всех новых домов. Каждый человек, конечно же, хочет минимально использовать энергию.

\subsection{Российские проекты Умных городов}
В России за последние годы появился ряд проектов по интеллектуализации городских сервисов. В основном это умное управление
в трех сферах: электроэнергия, транспортные потоки, общественная
безопасность.\cite{smartcity}

\subsubsection{ЦОД «Омский»}
Вокруг центра обработки данных «Омский» предполагается разместить промышленно-логистический парк (IT-кластер, высокотехнологичный сельскохозяйственный комплекс, производственные и лабораторные помещения), объекты жилой и социальной недвижимости (зона жилой застройки на 14 тыс. человек), административно-деловой центр и другие
объекты. Технологии Умного города появятся в разработках самих резидентов кластера – IT-компаний и стартапов в сфере облачных технологий. Подобно японским городам, «Омский» ставит целью перейти
на самоокупаемость, энерго- и даже отчасти продуктовую независимость: например, самостоятельно генерировать электричество, а его избытки передавать на содержание теплиц.

\subsubsection{Ильинское-Усово}
Проектировщики ЖК «Ильинское-Усово» внедряют умные городские технологии уже на первой очереди строительства микрорайона: это новые материалы, Интернет вещей в части ЖКХ, безопасности и транспорта. ГК «Мортон» определяет 15 направлений развития Умного города, и отдает предпочтение решениям, которые охватывают сразу несколько из них. На фоне базовых технологий (интеграция транспортной и инженерной систем,
энергосбережение, видеоаналитика и пр.) выделяется уникальная для России система недорогих решений в области медицины. Сама «Мортон» инвестирует в ряд городских технологий, которые впоследствии тиражирует.

\subsubsection{Владивосток}
Во Владивостоке работает Единая дежурная диспетчерская служба по управлению транспортом в реальном времени, а также автоматизированная система управления уличным освещением. «Ростелеком» внедряет в городе систему экстренного вызова оперативных служб и модернизированную систему оповещения.
\subsubsection{Белгород}
В Белгороде установлены Умное освещение и датчики на распределительных сетях, которые минимизируют последствия аварий.
\subsubsection{Екатеринбург}
Екатеринбург планирует создать интеллектуальную энергосеть к 2030 году. Столько займет модернизация существующих и построение новых энергообъектов с учетом требований Smart Grid, в том числе внедрение транспортных средств на электротяге, перевод объектов – потребителей электроэнергии в режим ее генерации.

\subsection{Умный город будет создан в Обнинске}
В Обнинске в рамках федеральной программы <<Цифровая экономика РФ>> будет создан Умный город. Об этом было заявлено калужским губернатором Анатолием Артамоновым на инвестиционном форуме в Сочи в феврале 2018 года.\cite{smartcityObn}
Выступая на сессии форума, калужский губернатор Анатолий Артамонов акцентировал внимание собравшихся, что для реализации данных проектов, необходимо, в первую очередь, совершенствовать инфраструктуру. Прежде всего, это касается городов, где она формировалась давно. По мнению губернатора, решение данной задачи возможно только при условии реализации единой госполитики. 

<<Это повлечет за собой большие расходы, но данное направление необходимо включить в число приоритетных>>, - предложил Артамонов. Что касается Обнинска, то создание Умного города входит в стратегию социально-экономического развития наукограда на ближайшие 9 лет. 

<<Тема новая, но очень перспективная. Нам нужно стремиться к тому, чтобы в будущем не только крупные города, но и небольшие населенные пункты могли внедрить у себя Умные технологии>>, - отметил губернатор. 

Ключевыми направлениями Умного города, охватывающими все виды социально - экономической деятельности городов, являются: «Умная экономика», «Умная мобильность», «Умная среда», «Умные люди», «Умная безопасность», «Умная медицина», «Умное проживание» и «Умное управление». 

В целом, проект можно считать достаточно молодым. Находится он на стадии старта. Но уже успел завоевать доверие и признанность не в одной стране мира. Масштаб и продвижение проекта в разных точках мира зависят, в основном, от времени начала его ввода в пользование. 

Каждый житель города имеет право принимать участие в развитии проекта, предлагать идеи и проявлять интерес, к таким деталям, на которые, по его мнению, стоит обратить внимание. 